% math-environments.tex
% LaTeX definitions for mathematical environments in PDF output

% Required packages
\usepackage{amsthm}
\usepackage{thmtools}
\usepackage{mdframed}
\usepackage{xcolor}

% Define colors for different environments
\definecolor{theoremblue}{RGB}{0, 102, 204}
\definecolor{definitiongray}{RGB}{102, 102, 102}
\definecolor{lemmaorange}{RGB}{255, 153, 0}
\definecolor{propositionpurple}{RGB}{153, 0, 204}
\definecolor{corollarygreen}{RGB}{0, 204, 102}
\definecolor{exampleyellow}{RGB}{204, 204, 0}

% Define theorem styles
\declaretheoremstyle[
    headfont=\bfseries\sffamily\color{theoremblue},
    bodyfont=\normalfont,
    mdframed={
        linewidth=2pt,
        rightline=false,
        topline=false,
        bottomline=false,
        linecolor=theoremblue,
        backgroundcolor=theoremblue!5
    }
]{theoremstyle}

\declaretheoremstyle[
    headfont=\bfseries\sffamily\color{definitiongray},
    bodyfont=\normalfont,
    mdframed={
        linewidth=1pt,
        linecolor=definitiongray,
        backgroundcolor=gray!5
    }
]{definitionstyle}

\declaretheoremstyle[
    headfont=\bfseries\sffamily\color{lemmaorange},
    bodyfont=\normalfont,
    mdframed={
        linewidth=2pt,
        rightline=false,
        topline=false,
        bottomline=false,
        linecolor=lemmaorange,
        backgroundcolor=lemmaorange!5
    }
]{lemmastyle}

\declaretheoremstyle[
    headfont=\bfseries\sffamily\color{propositionpurple},
    bodyfont=\normalfont,
    mdframed={
        linewidth=2pt,
        rightline=false,
        topline=false,
        bottomline=false,
        linecolor=propositionpurple,
        backgroundcolor=propositionpurple!5
    }
]{propositionstyle}

\declaretheoremstyle[
    headfont=\bfseries\sffamily\color{corollarygreen},
    bodyfont=\normalfont,
    mdframed={
        linewidth=2pt,
        rightline=false,
        topline=false,
        bottomline=false,
        linecolor=corollarygreen,
        backgroundcolor=corollarygreen!5
    }
]{corollarystyle}

\declaretheoremstyle[
    headfont=\bfseries\sffamily\color{exampleyellow},
    bodyfont=\itshape,
    mdframed={
        linewidth=2pt,
        rightline=false,
        topline=false,
        bottomline=false,
        linecolor=exampleyellow,
        backgroundcolor=exampleyellow!5
    }
]{examplestyle}

% Declare theorem environments
\declaretheorem[style=theoremstyle,numberwithin=section]{theorem}
\declaretheorem[style=definitionstyle,numberwithin=section]{definition}
\declaretheorem[style=lemmastyle,numberwithin=section]{lemma}
\declaretheorem[style=propositionstyle,numberwithin=section]{proposition}
\declaretheorem[style=corollarystyle,numberwithin=section]{corollary}
\declaretheorem[style=examplestyle,numberwithin=section]{example}

% Custom proof environment with QED symbol
\renewenvironment{proof}[1][\proofname]{%
  \par
  \pushQED{\qed}%
  \normalfont \topsep6\p@\@plus6\p@\relax
  \trivlist
  \item[\hskip\labelsep
        \itshape
    #1\@addpunct{.}]\ignorespaces
}{%
  \popQED\endtrivlist\@endpefalse
}

% Define QED symbol
\renewcommand{\qedsymbol}{$\square$}

% Hyperref support for cross-references
\usepackage{hyperref}
\hypersetup{
    colorlinks=true,
    linkcolor=blue,
    citecolor=blue,
    urlcolor=blue
}

% Cleveref support for automatic reference formatting
\usepackage{cleveref}
\crefname{theorem}{Theorem}{Theorems}
\crefname{definition}{Definition}{Definitions}
\crefname{lemma}{Lemma}{Lemmas}
\crefname{proposition}{Proposition}{Propositions}
\crefname{corollary}{Corollary}{Corollaries}
\crefname{example}{Example}{Examples}