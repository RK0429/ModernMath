% Options for packages loaded elsewhere
% Options for packages loaded elsewhere
\PassOptionsToPackage{unicode}{hyperref}
\PassOptionsToPackage{hyphens}{url}
\PassOptionsToPackage{dvipsnames,svgnames,x11names}{xcolor}
%
\documentclass[
  11pt,
  a4paper,
]{article}
\usepackage{xcolor}
\usepackage[margin=1in]{geometry}
\usepackage{amsmath,amssymb}
\setcounter{secnumdepth}{-\maxdimen} % remove section numbering
\usepackage{iftex}
\ifPDFTeX
  \usepackage[T1]{fontenc}
  \usepackage[utf8]{inputenc}
  \usepackage{textcomp} % provide euro and other symbols
\else % if luatex or xetex
  \usepackage{unicode-math} % this also loads fontspec
  \defaultfontfeatures{Scale=MatchLowercase}
  \defaultfontfeatures[\rmfamily]{Ligatures=TeX,Scale=1}
\fi
\usepackage{lmodern}
\ifPDFTeX\else
  % xetex/luatex font selection
\fi
% Use upquote if available, for straight quotes in verbatim environments
\IfFileExists{upquote.sty}{\usepackage{upquote}}{}
\IfFileExists{microtype.sty}{% use microtype if available
  \usepackage[]{microtype}
  \UseMicrotypeSet[protrusion]{basicmath} % disable protrusion for tt fonts
}{}
\makeatletter
\@ifundefined{KOMAClassName}{% if non-KOMA class
  \IfFileExists{parskip.sty}{%
    \usepackage{parskip}
  }{% else
    \setlength{\parindent}{0pt}
    \setlength{\parskip}{6pt plus 2pt minus 1pt}}
}{% if KOMA class
  \KOMAoptions{parskip=half}}
\makeatother
% Make \paragraph and \subparagraph free-standing
\makeatletter
\ifx\paragraph\undefined\else
  \let\oldparagraph\paragraph
  \renewcommand{\paragraph}{
    \@ifstar
      \xxxParagraphStar
      \xxxParagraphNoStar
  }
  \newcommand{\xxxParagraphStar}[1]{\oldparagraph*{#1}\mbox{}}
  \newcommand{\xxxParagraphNoStar}[1]{\oldparagraph{#1}\mbox{}}
\fi
\ifx\subparagraph\undefined\else
  \let\oldsubparagraph\subparagraph
  \renewcommand{\subparagraph}{
    \@ifstar
      \xxxSubParagraphStar
      \xxxSubParagraphNoStar
  }
  \newcommand{\xxxSubParagraphStar}[1]{\oldsubparagraph*{#1}\mbox{}}
  \newcommand{\xxxSubParagraphNoStar}[1]{\oldsubparagraph{#1}\mbox{}}
\fi
\makeatother


\usepackage{longtable,booktabs,array}
\usepackage{calc} % for calculating minipage widths
% Correct order of tables after \paragraph or \subparagraph
\usepackage{etoolbox}
\makeatletter
\patchcmd\longtable{\par}{\if@noskipsec\mbox{}\fi\par}{}{}
\makeatother
% Allow footnotes in longtable head/foot
\IfFileExists{footnotehyper.sty}{\usepackage{footnotehyper}}{\usepackage{footnote}}
\makesavenoteenv{longtable}
\usepackage{graphicx}
\makeatletter
\newsavebox\pandoc@box
\newcommand*\pandocbounded[1]{% scales image to fit in text height/width
  \sbox\pandoc@box{#1}%
  \Gscale@div\@tempa{\textheight}{\dimexpr\ht\pandoc@box+\dp\pandoc@box\relax}%
  \Gscale@div\@tempb{\linewidth}{\wd\pandoc@box}%
  \ifdim\@tempb\p@<\@tempa\p@\let\@tempa\@tempb\fi% select the smaller of both
  \ifdim\@tempa\p@<\p@\scalebox{\@tempa}{\usebox\pandoc@box}%
  \else\usebox{\pandoc@box}%
  \fi%
}
% Set default figure placement to htbp
\def\fps@figure{htbp}
\makeatother





\setlength{\emergencystretch}{3em} % prevent overfull lines

\providecommand{\tightlist}{%
  \setlength{\itemsep}{0pt}\setlength{\parskip}{0pt}}






\usepackage{amsmath,amssymb,amsthm}
\usepackage{mathtools}
\usepackage{unicode-math}
\makeatletter
\@ifpackageloaded{caption}{}{\usepackage{caption}}
\AtBeginDocument{%
\ifdefined\contentsname
  \renewcommand*\contentsname{Table of contents}
\else
  \newcommand\contentsname{Table of contents}
\fi
\ifdefined\listfigurename
  \renewcommand*\listfigurename{List of Figures}
\else
  \newcommand\listfigurename{List of Figures}
\fi
\ifdefined\listtablename
  \renewcommand*\listtablename{List of Tables}
\else
  \newcommand\listtablename{List of Tables}
\fi
\ifdefined\figurename
  \renewcommand*\figurename{Figure}
\else
  \newcommand\figurename{Figure}
\fi
\ifdefined\tablename
  \renewcommand*\tablename{Table}
\else
  \newcommand\tablename{Table}
\fi
}
\@ifpackageloaded{float}{}{\usepackage{float}}
\floatstyle{ruled}
\@ifundefined{c@chapter}{\newfloat{codelisting}{h}{lop}}{\newfloat{codelisting}{h}{lop}[chapter]}
\floatname{codelisting}{Listing}
\newcommand*\listoflistings{\listof{codelisting}{List of Listings}}
\makeatother
\makeatletter
\makeatother
\makeatletter
\@ifpackageloaded{caption}{}{\usepackage{caption}}
\@ifpackageloaded{subcaption}{}{\usepackage{subcaption}}
\makeatother
\usepackage{bookmark}
\IfFileExists{xurl.sty}{\usepackage{xurl}}{} % add URL line breaks if available
\urlstyle{same}
\hypersetup{
  pdftitle={About This Project},
  colorlinks=true,
  linkcolor={blue},
  filecolor={Maroon},
  citecolor={Blue},
  urlcolor={Blue},
  pdfcreator={LaTeX via pandoc}}


\title{About This Project}
\author{}
\date{}
\begin{document}
\maketitle


\section{About the Mathematics Knowledge Graph
Wiki}\label{about-the-mathematics-knowledge-graph-wiki}

\subsection{Vision}\label{vision}

This project aims to create a comprehensive, queryable knowledge graph
that represents mathematical knowledge as interconnected concepts. By
combining human-readable documentation with machine-readable semantic
data, we enable both learning and automated reasoning about mathematics.

\subsection{Technical Architecture}\label{technical-architecture}

\subsubsection{Content Layer (Quarto)}\label{content-layer-quarto}

\begin{itemize}
\tightlist
\item
  Human-readable mathematical content
\item
  LaTeX equation support
\item
  Cross-referencing system
\item
  Structured metadata in YAML
\end{itemize}

\subsubsection{Semantic Layer (RDF/OWL)}\label{semantic-layer-rdfowl}

\begin{itemize}
\tightlist
\item
  Formal ontology defining mathematical concepts
\item
  RDF triples representing relationships
\item
  SPARQL endpoint for querying
\item
  Linked Data publication
\end{itemize}

\subsubsection{Verification Layer (Lean
4)}\label{verification-layer-lean-4}

\begin{itemize}
\tightlist
\item
  Formal mathematical proofs
\item
  Dependency tracking
\item
  Integration with mathlib4
\item
  Verification of logical consistency
\end{itemize}

\subsubsection{Visualization Layer}\label{visualization-layer}

\begin{itemize}
\tightlist
\item
  Interactive graph visualizations
\item
  Dependency diagrams
\item
  Concept maps
\item
  Learning path visualization
\end{itemize}

\subsection{Ontology Structure}\label{ontology-structure}

Our knowledge graph uses the following core node types:

\begin{itemize}
\tightlist
\item
  \textbf{Axiom}: Fundamental assumptions taken as true
\item
  \textbf{Definition}: Formal definitions of mathematical concepts
\item
  \textbf{Theorem}: Proven mathematical statements (including lemmas,
  propositions, corollaries)
\item
  \textbf{Example}: Concrete instances illustrating concepts
\end{itemize}

And the following relationships:

\begin{itemize}
\tightlist
\item
  \textbf{uses/dependsOn}: Logical dependencies between concepts
\item
  \textbf{hasExample}: Links concepts to their examples
\item
  \textbf{generalizes/specializes}: Hierarchical relationships
\item
  \textbf{implies}: Logical implications
\end{itemize}

\subsection{Development Status}\label{development-status}

This project is under active development. See our
\href{https://github.com/RK0429/ModernMath}{GitHub repository} for the
latest updates.

\subsection{Contact}\label{contact}

For questions or contributions, please open an issue on our GitHub
repository.




\end{document}
